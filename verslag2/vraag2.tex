\documentclass[a4paper,kulak]{kulakarticle} %options: kul or kulak (default)

\usepackage[utf8]{inputenc}
\usepackage[dutch]{babel}

\date{Academiejaar 2018 -- 2019}
\address{
  Naam van de opleiding \\
  Naam van het vak \\
  Naam van de docent/begeleiders}
\title{Titel van het document}
\author{Naam van de auteurs}


\begin{document}

\maketitle

\section{Model voor de huurprijs}
Op basis van relevante beschikbare gegevens hebben we geprobeerd een model op te stellen om de prijs van een Airbnb-verblijf zo goed mogelijk te voorspellen. Eerst moesten we daarvoor kijken welke gegevens er relevant zouden zijn voor ons model. Daarna controleerden we of het nodig was om bij bepaalde variabelen een transformatie toe te passen. We bekeken enkele modellen met en zonder transformaties en tot slot bepaalden we het beste model om de prijs te voorspellen.
\subsection{Selectie relevante gegevens}
Om variabelen te selecteren als een relevante variabele voor ons model, keken we naar hoe de prijs zich verhoudt ten opzichte van de variabele. Eerst en vooral merken we op dat "'id"', "'name"', "'host\_id"' en "'host\_name"' niets te maken kunnen hebben met de prijs, deze gegevens laten we dus achterwege. De variabele "'latitude"' blijkt wel een in vloed te hebben op de prijs. Als we kijken naar de waarden van de breedtegraden dan zien we dat we deze kunnen indelen in de drie steden. We gaan in plaats van de breedtegraad in ons model te gebruiken, gebruik maken van de variabele "'city"'. "'Longitude"' wordt om dezelfde reden niet gebruikt in het model, deze variabele deelt de data op in twee delen namelijk als eerste deel Gent en als tweede deel Brussel en Antwerpen. Vervolgens zien we ook dat variabelen "'room\_type"', "'minimum\_nights"', "'number\_of\_reviews"', "'last\_review"', "'reviews\_per\_month"', "'availability\_365"' en "'calculated\_host\_listings\_count"' relevant kunnen zijn voor ons model. \\

Voor de variabelen die iets zeggen over de reviews hebben ook eens gekeken naar de correlatie tussen deze variabelen. We vonden dat er een sterke correlatie bestaat tussen "'reviews\_per\_month"' en "'number\_of\_reviews"', ook is er een correlatie tussen "'reviews\_per\_month"' en "'last\_review"'. Om deze reden gaan we zeker al twee verschillende modellen opstellen waarbij enerzijds gewerkt wordt met alle drie de variabelen en anderzijds enkel met de variabele "'reviews\_per\_month"'.
\subsection{Transformaties}
In deze sectie bekijken we of er transformaties zijn die ervoor kunnen zorgen dat we een beter model verkrijgen om de prijs te voorspellen. 
\begin{tabular}{|l|l|}
	\hline 
	\textbf{Variabele} & \textbf{Transformatie} \\ 
	\hline 
	room\_type & / \\ 
	\hline 
	city & / \\ 
	\hline 
	price & $\frac{(price^{(-0.25)}) - 1}{-0.25}$ \\ 
	\hline 
	minimum\_nights & $\frac{(minimum\_nights^{(-0.67)}) - 1}{-0.67}$ \\ 
	\hline 
	number\_of\_reviews &  log10(number\_of\_reviews)\\ 
	\hline 
	last\_review & log10(last\_review + 1)\\
	\hline
	reviews\_per\_month &log10(reviews\_per\_month) \\
	\hline
	calculated\_host\_listings\_count & $\frac{(calculated\_host\_listings\_count^{(-1)}) - 1}{-1}$\\
	\hline
	availability\_365 & /\\ \hline
\end{tabular} 
\subsection{Modellen}

\end{document}
